\chapter{Introduction}\pagenumbering{arabic} %Switch to normal numbers

Phytoplankton are photosynthesizing components of the plankton community that inhabit the upper light lit layers of water in almost all oceans and bodies of fresh water. Phytoplankton obtain energy from the Sun through the process, photosynthesis. It is estimated that phytoplankton account for around half of all photosynthetic activity on Earth, \cite{halfO2} yet only account for 1\% of Earth's biomass. \cite{bidle2004cell} 
\\\\
As well as providing half of our atmospheres oxygen, phytoplankton remove large amounts of carbon dioxide from our atmosphere. Through photosynthesis, phytoplankton turn carbon dioxide from our atmosphere into carbonates. Once this carbon is fixed into soft or hard tissue, the organisms either stay in the upper light lit regions to again be used in photosynthesis or once they die, begin to sink to the bottom of the ocean floor. The fixed carbon is then either decomposed by bacteria on the way down or once on the sea floor is remineralized to be used again in primary production. The particles that escape these processes entirely are sequestered at the bottom of the ocean in the ground sediment and may remain there for millions of years. It is this sequestered carbon that is responsible for lowering atmospheric carbon dioxide.
\\\\
It is these important characteristics that give motivation to study the health of phytoplankton and more importantly to understand what can be done to ensure longevity of this vital species. 

\section{Marine Heat Waves and their Impacts}

\begin{quote}
    An extreme event is an event that is rare at a particular place and time of year. Definitions of 'rare' vary, but an extreme event would normally be as rare as or rarer than the 10th or 90th percentile of a probability density function estimated from observations. \hfill\cite[p.~6-8]{IPCC2019}
\end{quote}

Sea temperatures recorded to be extreme for their respective location for a period of five or more consecutive days are so-called marine heat waves (MHWs). Sucessive heatwaves with gaps of two days or less are considered part of the same event. \cite{MWHdef} Marine heatwaves and their effects on natural, physical and socio-economic systems have been well documented in all ocean basins over the last two decades. A notable example is the Northeast Pacific 2013-2015 MWH where extremely high temperatures were linked to the presence of elevated levels of a natural toxin domoic acid (DA) in razor clams \cite{toxinheat}. 
\\\\
Domoic acid is a kainic acid-type neurotoxin that is produced by harmful phytoplankton species. After a large-scale algal bloom in late 2015 levels of DA along the west coast of North America were deemed above safe thresholds. This in turn lead to the closure of recreational razor clam harvest and Dungeness crab fisheries in a variety of locations along the west coast of North America. "Fisheries of the United States 2015" report showed a decreased value of Dungeness crab fished of almost a \$100 million compared to the previous year. \cite{2015Cost} In addition to negative economic impacts, toxic events can lead to the mass death of marine mammals and can pose potential public health issues for humans. Human consumption of shellfish with high levels of DA can cause a neurological disorder called domoic acid poisoning. 

\section{Changes to Marine Heat Waves under Climate Change}
When talking about certainty, we will use the following terms to express the assessed likelihood of an outcome or a result: Virtually certain 99-100\% probability, Very likely 90-100\%, Likely 66-100\%, About as likely as not 33-66\%, Unlikely 0-33\%, Very unlikely 0-10\% and Exceptionally unlikely 0-1\%
\\\\
The International Governmental Panel on Climate Change (IPCC) WGI AR5 report concluded that it is \textit{virtually} certain that the global ocean temperature in the upper few hundred metres has increased from 1971-2010 \cite{AR5}, and is projected to further increase during the twenty-first century. \cite{collins2013long} Concurrent with the long-term increase in the upper ocean temperatures, MHWs have become more frequent, extensive and intense. \cite{1,2,3}
\\\\
Analysis of satellite daily sea surface temperature data from 1982-2016 shows that the number of MHW days above the 99th percentile, between 1982 and 2016 has doubled globally from around 2.5 to 5 heatwave-days per year \cite{1,2}. During the same period the maximum intensity of MHWs has increased by $0.3^\circ$C and the spatial extent by 66\%. \cite{1} The observed trend towards more frequent, extensive and intense MHWs, defined to a fixed baseline period, is \textit{very likely} due to anthropogenic increase in mean ocean temperatures. \cite{1,2,3} As climate models project a long-term increase in ocean temperatures over the 21st century \cite{collins2013long} a further increase in the probability of of MHWs under continued global warming can be expected. \cite{IPCC2019}
\\\\